% -*- root: ../thesis.tex -*-
%%%%%%%%%%%%%%%%%%%%----------%%%%%%%%%%%%%%%%%%%%
%%%
% 	Author: Boris Kourtoukov
%   Project: Digital Futures Thesis Document
%   Title: Periphery
% 	File: Outcomes
%%%
%%%%%%%%%%%%--------------------------%%%%%%%%%%%%

This research and exploration project looks at the possibilities within what we wear on the body in our near future. With the direct intention to provide an open set of findings, designs and physical artifacts that can be iterated on and used by a wider developer community. It is the hope of this project that it will now serve to inspire a leap into the future as proposed by Cyberpunk ideas of what is visually pleasing and technologically relevant. Within Digital Futures the major areas of focus for Periphery have been on the design and scientific research here at OCADU.  

\begin{marginfigure}
  \includegraphics{core.jpg}
  \caption{The current prototype, light sensor, sound and an indicator LED. Interfaced to the body using rare earth magnets.}
  \label{fig:currentprototype}
\end{marginfigure}

\section{Possibilities of the prototype}
There are a number of user centric scenarios that can be seen in Periphery as a wearable ambient sensor network.

\mqt{What can be achieved on a small scale?}
\subsection{The personal climate tweaker}\label{subsec:personalclimate}
Many users can pick up and use the near raw data of periphery now. The main application would be a clear analysis of the overall days and weeks within our lives. Periphery doesn't carry any location specific data, but what it shows is a clear time line of sensor events through each day and week. Taking just the light data as an example, being able to see your lighting conditions over time can yield adjustments to your daily habits and locations. For example it might be much darker in the users office then they realize, but by seeing a large light level drop whenever they enter their office space they can correlate that lack of light to mood drops and difficulties with focus. We tend to not register our surroundings on a minute by minute basis for longer than a few moments, unless something truly extreme is happening at the time. Periphery can provide a window in to the world we forget, and let us tweak our future experiences based on that data.

\mqt{What could be an example of a broader project?}
\subsection{Creating a value driven data collection system}
Currently the trend of user generated data is that it is exchanged for a free service. Google provide its search for free in exchange for targeted advertisements and using your habits to rank web pages. Periphery could provide a new set of data that a company like Google would be interested in. This time though the user can offer to sell that data to Google, creating incentive to wear the system as well as providing a passive source of income to anyone who wishes to participate. Google in turn can use this to calculate more precise weather, visualize the atmosphere of spaces in real time and create a new realm of analytics. 

\newthought{These possibilities} provide a glimpse into what is possible with Periphery, but there are many areas left to be explored. The open nature of the project should be taken as an invitation for that exploration, and as a way to see what can be done with this data and how it can be enhanced.