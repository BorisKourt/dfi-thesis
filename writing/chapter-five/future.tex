% -*- root: ../thesis.tex -*-
%%%%%%%%%%%%%%%%%%%%----------%%%%%%%%%%%%%%%%%%%%
%%%
% 	Author: Boris Kourtoukov
%   Project: Digital Futures Thesis Document
%   Title: Periphery
% 	File: Future Work
%%%
%%%%%%%%%%%%--------------------------%%%%%%%%%%%%

There are two clear areas that should be built upon from this base prototype.

\subsection{Style Guide}

The style guide should be fleshed out with material exploration, although the modules were completed with a rudimentary implementation of the concept art it is not close enough to that art to provide the desired impact. As mentioned earlier 3D printing is becoming more and more robust, and further exploration of Chenthooran's artwork in that medium is the best way to proceed with the physical design. This could also tie in to seeing whether enclosing the electronics entirely would provide a better user experience or not. Especially since the hands(and cords) free nature of the devices now already plays well to the experience of using them.

\subsection{Demo applications}

The next step on the technical side is to use this robust and reliable API to create some simple yet visually compelling mobile applications. These applications should aim to solve a basic problem, like the \nameref{subsec:personalclimate}. And to leave as much interpretation to the user as possible, in order to generate discussion on how they use the data steam themselves. Initially this could involve a static sample chunk of data lasting a week or a month, upon which developers can perform queries and computations. 

\newthought{With that said} there are many further areas that can be explored with the framework set by Periphery. The openness of the project is intended to provide a way for people to come up with their own solutions without being tied into proprietary or rapidly obsolete systems.