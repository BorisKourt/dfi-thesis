% -*- root: ../thesis.tex -*-
%%%%%%%%%%%%%%%%%%%%----------%%%%%%%%%%%%%%%%%%%%
%%%
% 	Author: Boris Kourtoukov
%   Project: Digital Futures Thesis Document
%   Title: Periphery
% 	File: Background
%%%
%%%%%%%%%%%%--------------------------%%%%%%%%%%%%

Tests of practical wearability of modular objects have been done by Gemperle, Kasabach, Stivoric, Bauer, and Martin in the ``Design for Wearability'' paper \citep{Gemperle98}. The article provided valuable insight for the placement of wearable objects, or pods, on the body as relative to their comfort and impact on the form. The participants that wore these pods generally had positive remarks about their overall impact. As each part of the body requires different considerations about movement, shape, and weight the guidelines in this paper have greatly benefited with the placement of modules.

In their 2010 paper ``Mobile Computing in the Context of Calm Technology'', Fiaidhi, Jinan, Chou, and Williams talk about not limiting the availability of information in mobile devices but rather let the users context dictate what is immediate \citep{Fiaidhi10}. This paper helped provide a better understanding of how to create these immediate experiences without detracting from the wealth of data that should be available if desired. And although the development of tools that interface with provided sensor data was not in the scope, this is an important read to any future application developers.

Another important paper is the ``Psychophysical Elements of Wearability'' \citep{Dunne07}. One of the points this article covers is that adaptation is a vital factor when considering wearability. There are major differences in the initial reaction to the responses after an adjustment period. When a foreign object is attached to the bodies we are so used to, it requires time for us to grow connected with it. This provided a helpful concept to keep in mind throughout, and connected with ``Design for Wearability'' \citep{Gemperle98} created a better picture of how to build comfortable wearables.