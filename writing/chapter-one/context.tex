% -*- root: ../thesis.tex -*-
%%%%%%%%%%%%%%%%%%%%----------%%%%%%%%%%%%%%%%%%%%
%%%
% 	Author: Boris Kourtoukov
%   Project: Digital Futures Thesis Document
%   Title: Periphery
% 	File: Context
%%%
%%%%%%%%%%%%--------------------------%%%%%%%%%%%%

\mqp{What is in the now?}
The field of wearable electronics is swiftly growing. Major industry giants like Google\sidenote{\url{http://google.com/glass}} and Samsung\sidenote{\url{http://samsung.com/gear}} are releasing products that are aimed to capitalize on the ubiquity of commonplace wearables like glasses and watches. As more and more companies enter this space wrists and eyes are becoming a very crowded market. And although this is greatly increasing the interest on the field it is important to note that, from this direction, there are not many examples of placing technology on other parts of the body.

\mqs{Is there a total lack of exploration?}
Research facilities like the Social Body Lab\sidenote{\url{http://research.ocadu.ca/socialbody/home}} are a host to projects that focus entirely on interfacing technology with many aspects and parts of the body. The Nudgeables Accessory Kit\sidenote{\url{http://research.ocadu.ca/socialbody/project/nudgeables-accessory-kit}} is an especially good example of this as it encourages users to integrate a barebone system into discrete and uncommon pieces of their garments. Students and workshop participants have integrated the Nudgeables into everything from ties to bras. 

A collaboration between professors at the Northwestern and Illinois Universities has yielded\sidenote{\url{https://vimeo.com/90902153}} a health monitor that presents another take on what it means to be wearable. The monitor is a fully flexible patch that can be placed directly on the skin and requires no further interfacing. 

Both of these projects step back from replicating existing gadgetry, and rather look at the how and where technology can be placed on the body. 

\mqt{What else connects into this?}
\newthought{Extension of the body} is a very curious piece of today's wearables puzzle. Part of the motivation behind this project revolves around a response to prosthesis research. There is a strong movement toward advancements within that field, many researchers have created sophisticated replicas of various parts of the human body. This progress is wondrous development for those that have lost full function of a part of their body, giving them hope and confidence in their future. In contrast there is less development in non intrusive augmentation. As stated earlier current wearable technology projects that reach the spotlight are but adjustments of the gadgets from the last centuries, and they limit the acceptance of augmenting a more diverse range of locations on the body. As a society, looking at what is popularized by consumers, we are not yet keen on exploring these other possibilities.

\mqs{What is the possible future here?}
The research and development of prosthesis devices far outpaces the exploration of non intrusive augmentation. It is possible that in the future parts of the body will be redeveloped in ways that they would be seen as far superior to the way those function unaided. At that precipice there will likely be the first humans that chose to remove or permanently adjust themselves in order to integrate these more advanced parts. This could very well happen before we are wholly aware of what is possible without such intrusive methods. One of the goals of this thesis project is to create a final aesthetic design that can be taken and explored by others in order to further the research into non invasive augmentation. It is important that we become aware of what is not only possible alongside our bodies but can become openly accepted in public, before we begin to permanently modify ourselves with technology. 

\mqt{Well that was embarrassingly lofty wasn't it?}
\newthought{This might be} seen as an unattainable goal for an undergraduate thesis, and it is, the attempt is nonetheless what is important to this project. The small piece of this problem that is approached here is the creation of a visual style guide for a wearable future that isn't afraid of soft augmentation.

