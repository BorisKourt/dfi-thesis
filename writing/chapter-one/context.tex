% -*- root: ../thesis.tex -*-
%%%%%%%%%%%%%%%%%%%%----------%%%%%%%%%%%%%%%%%%%%
%%%
% 	Author: Boris Kourtoukov
%   Project: Digital Futures Thesis Document
%   Title: Periphery
% 	File: Context
%%%
%%%%%%%%%%%%--------------------------%%%%%%%%%%%%

\mqp{What is in the now?}
The field of wearable electronics is swiftly growing. Major industry giants like Google\sidenote{\url{http://google.com/glass}} and Samsung\sidenote{\url{http://samsung.com/gear}} are releasing products that are aimed to capitalize on the ubiquity of commonplace wearables like glasses and watches. As more and more companies enter this space wrists and eyes are becoming a very crowded market. And although this is greatly increasing the interest on the field it is important to note that, from this direction, there are not many examples of placing technology on other parts of the body.

\mqs{Is there a total lack of exploration?}
Research facilities like the Social Body Lab\sidenote{\url{http://research.ocadu.ca/socialbody/home}} are a host to projects that focus entirely on interfacing technology with many aspects and parts of the body. The Nudgeables Accessory Kit\sidenote{\url{http://research.ocadu.ca/socialbody/project/nudgeables-accessory-kit}} is an especially good example of this as it encourages users to integrate a barebone system into discrete and uncommon pieces of their garments. Students and workshop participants have integrated the Nudgeables into everything from ties to bras. 

A collaboration between professors at the Northwestern and Illinois Universities has yielded\sidenote{\url{https://vimeo.com/90902153}} a health monitor that presents another take on what it means to be wearable. The monitor is a fully flexible patch that can be placed directly on the skin and requires no further interfacing. 

Both of these projects step back from replicating existing gadgetry, and rather look at the how and where technology can be placed on the body.