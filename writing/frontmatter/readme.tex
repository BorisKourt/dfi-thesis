% -*- root: ../thesis.tex -*-
%%%%%%%%%%%%%%%%%%%%----------%%%%%%%%%%%%%%%%%%%%
%%%
% 	Author: Boris Kourtoukov
%   Project: Digital Futures Thesis Document
%   Title: Periphery
% 	File: Readme
%%%
%%%%%%%%%%%%--------------------------%%%%%%%%%%%%

Although the sections in this document follow a clear hierarchy, the content within each section is distributed a little differently.

\newthought{Each subsection} breaks down into an answer and question paring, with the questions appearing in the margin and the answers appearing to the left. This is a reverse format for what has been done in books like The Little Schemer \cite{littleSchemer} and The Reasoned Schemer \cite{reasonedSchemer}, both books are for learning the fundamentals of functional and logic programming.

\mqs{Why, then, are questions and answers reversed?}
Although having a question followed directly by an answer works for The Little Schemer, the goal of that book is fundamentally different from this document. Their focus is on the learning benefits of the approach while here the focus is on depicting the background thought process. The flow of the paragraphs does not have to be disturbed by the explicit questions, rather, questions remain as a way to visualize the chain of reasoning throughout. This will hopefully provide a unique and interesting feel to this text.

\mqt{Isn't this for Digital Futures? (\footnotesize{\url{http://ocdau.ca/}})}
\newthought{Additionally} the text will introduce as much ``digital'' into this thesis paper as possible. Resources will be (hyper!)linked to their original sources, and internal links will also be present throughout. The document is available at its GitHub\sidenote{\url{https://github.com/boriskourt/dfi-thesis/}} repository. Thus if you are one of the few that will read this on recycled trees a more up-to-date \& functional version is always available.